% Options for packages loaded elsewhere
\PassOptionsToPackage{unicode}{hyperref}
\PassOptionsToPackage{hyphens}{url}
%
\documentclass[
]{article}
\usepackage{amsmath,amssymb}
\usepackage{iftex}
\ifPDFTeX
  \usepackage[T1]{fontenc}
  \usepackage[utf8]{inputenc}
  \usepackage{textcomp} % provide euro and other symbols
\else % if luatex or xetex
  \usepackage{unicode-math} % this also loads fontspec
  \defaultfontfeatures{Scale=MatchLowercase}
  \defaultfontfeatures[\rmfamily]{Ligatures=TeX,Scale=1}
\fi
\usepackage{lmodern}
\ifPDFTeX\else
  % xetex/luatex font selection
\fi
% Use upquote if available, for straight quotes in verbatim environments
\IfFileExists{upquote.sty}{\usepackage{upquote}}{}
\IfFileExists{microtype.sty}{% use microtype if available
  \usepackage[]{microtype}
  \UseMicrotypeSet[protrusion]{basicmath} % disable protrusion for tt fonts
}{}
\makeatletter
\@ifundefined{KOMAClassName}{% if non-KOMA class
  \IfFileExists{parskip.sty}{%
    \usepackage{parskip}
  }{% else
    \setlength{\parindent}{0pt}
    \setlength{\parskip}{6pt plus 2pt minus 1pt}}
}{% if KOMA class
  \KOMAoptions{parskip=half}}
\makeatother
\usepackage{xcolor}
\usepackage[margin=1in]{geometry}
\usepackage{color}
\usepackage{fancyvrb}
\newcommand{\VerbBar}{|}
\newcommand{\VERB}{\Verb[commandchars=\\\{\}]}
\DefineVerbatimEnvironment{Highlighting}{Verbatim}{commandchars=\\\{\}}
% Add ',fontsize=\small' for more characters per line
\usepackage{framed}
\definecolor{shadecolor}{RGB}{248,248,248}
\newenvironment{Shaded}{\begin{snugshade}}{\end{snugshade}}
\newcommand{\AlertTok}[1]{\textcolor[rgb]{0.94,0.16,0.16}{#1}}
\newcommand{\AnnotationTok}[1]{\textcolor[rgb]{0.56,0.35,0.01}{\textbf{\textit{#1}}}}
\newcommand{\AttributeTok}[1]{\textcolor[rgb]{0.13,0.29,0.53}{#1}}
\newcommand{\BaseNTok}[1]{\textcolor[rgb]{0.00,0.00,0.81}{#1}}
\newcommand{\BuiltInTok}[1]{#1}
\newcommand{\CharTok}[1]{\textcolor[rgb]{0.31,0.60,0.02}{#1}}
\newcommand{\CommentTok}[1]{\textcolor[rgb]{0.56,0.35,0.01}{\textit{#1}}}
\newcommand{\CommentVarTok}[1]{\textcolor[rgb]{0.56,0.35,0.01}{\textbf{\textit{#1}}}}
\newcommand{\ConstantTok}[1]{\textcolor[rgb]{0.56,0.35,0.01}{#1}}
\newcommand{\ControlFlowTok}[1]{\textcolor[rgb]{0.13,0.29,0.53}{\textbf{#1}}}
\newcommand{\DataTypeTok}[1]{\textcolor[rgb]{0.13,0.29,0.53}{#1}}
\newcommand{\DecValTok}[1]{\textcolor[rgb]{0.00,0.00,0.81}{#1}}
\newcommand{\DocumentationTok}[1]{\textcolor[rgb]{0.56,0.35,0.01}{\textbf{\textit{#1}}}}
\newcommand{\ErrorTok}[1]{\textcolor[rgb]{0.64,0.00,0.00}{\textbf{#1}}}
\newcommand{\ExtensionTok}[1]{#1}
\newcommand{\FloatTok}[1]{\textcolor[rgb]{0.00,0.00,0.81}{#1}}
\newcommand{\FunctionTok}[1]{\textcolor[rgb]{0.13,0.29,0.53}{\textbf{#1}}}
\newcommand{\ImportTok}[1]{#1}
\newcommand{\InformationTok}[1]{\textcolor[rgb]{0.56,0.35,0.01}{\textbf{\textit{#1}}}}
\newcommand{\KeywordTok}[1]{\textcolor[rgb]{0.13,0.29,0.53}{\textbf{#1}}}
\newcommand{\NormalTok}[1]{#1}
\newcommand{\OperatorTok}[1]{\textcolor[rgb]{0.81,0.36,0.00}{\textbf{#1}}}
\newcommand{\OtherTok}[1]{\textcolor[rgb]{0.56,0.35,0.01}{#1}}
\newcommand{\PreprocessorTok}[1]{\textcolor[rgb]{0.56,0.35,0.01}{\textit{#1}}}
\newcommand{\RegionMarkerTok}[1]{#1}
\newcommand{\SpecialCharTok}[1]{\textcolor[rgb]{0.81,0.36,0.00}{\textbf{#1}}}
\newcommand{\SpecialStringTok}[1]{\textcolor[rgb]{0.31,0.60,0.02}{#1}}
\newcommand{\StringTok}[1]{\textcolor[rgb]{0.31,0.60,0.02}{#1}}
\newcommand{\VariableTok}[1]{\textcolor[rgb]{0.00,0.00,0.00}{#1}}
\newcommand{\VerbatimStringTok}[1]{\textcolor[rgb]{0.31,0.60,0.02}{#1}}
\newcommand{\WarningTok}[1]{\textcolor[rgb]{0.56,0.35,0.01}{\textbf{\textit{#1}}}}
\usepackage{graphicx}
\makeatletter
\def\maxwidth{\ifdim\Gin@nat@width>\linewidth\linewidth\else\Gin@nat@width\fi}
\def\maxheight{\ifdim\Gin@nat@height>\textheight\textheight\else\Gin@nat@height\fi}
\makeatother
% Scale images if necessary, so that they will not overflow the page
% margins by default, and it is still possible to overwrite the defaults
% using explicit options in \includegraphics[width, height, ...]{}
\setkeys{Gin}{width=\maxwidth,height=\maxheight,keepaspectratio}
% Set default figure placement to htbp
\makeatletter
\def\fps@figure{htbp}
\makeatother
\setlength{\emergencystretch}{3em} % prevent overfull lines
\providecommand{\tightlist}{%
  \setlength{\itemsep}{0pt}\setlength{\parskip}{0pt}}
\setcounter{secnumdepth}{-\maxdimen} % remove section numbering
\usepackage{booktabs}
\usepackage{longtable}
\usepackage{array}
\usepackage{multirow}
\usepackage{wrapfig}
\usepackage{float}
\usepackage{colortbl}
\usepackage{pdflscape}
\usepackage{tabu}
\usepackage{threeparttable}
\usepackage{threeparttablex}
\usepackage[normalem]{ulem}
\usepackage{makecell}
\usepackage{xcolor}
\ifLuaTeX
  \usepackage{selnolig}  % disable illegal ligatures
\fi
\usepackage{bookmark}
\IfFileExists{xurl.sty}{\usepackage{xurl}}{} % add URL line breaks if available
\urlstyle{same}
\hypersetup{
  pdftitle={Tarifa Zero no Brasil},
  pdfauthor={Aline Maria e João Lucas},
  hidelinks,
  pdfcreator={LaTeX via pandoc}}

\title{Tarifa Zero no Brasil}
\author{Aline Maria e João Lucas}
\date{2024-09-06}

\begin{document}
\maketitle

{
\setcounter{tocdepth}{2}
\tableofcontents
}
\begin{Shaded}
\begin{Highlighting}[]
\FunctionTok{library}\NormalTok{(pacman)}
\end{Highlighting}
\end{Shaded}

\begin{verbatim}
## Warning: package 'pacman' was built under R version 4.3.3
\end{verbatim}

\begin{Shaded}
\begin{Highlighting}[]
\NormalTok{pacman}\SpecialCharTok{::}\FunctionTok{p\_load}\NormalTok{(dplyr, psych, car, MASS, DescTools, QuantPsyc, ggplot2, readxl, kableExtra)}
\end{Highlighting}
\end{Shaded}

\section{Objeto: Tarifa Zero no
Brasil}\label{objeto-tarifa-zero-no-brasil}

No Brasil, o sistema de transporte coletivo urbano é um serviço público,
sendo comumente operado por empresas privadas a partir de concessões ou
permissões, municipais ou distritais (Schiaffino; Toledo; Ribeiro,
2015). Dessa forma, apesar do governo gerir esse serviço, o seu
funcionamento está suscetível às demandas e interesses de grandes
empresários do transporte. Sendo assim, o valor das tarifas a ser pago
pelo próprio usuário tende a ser, muitas vezes, insustentável para uma
parcela significativa das pessoas que dependem desse serviço para se
locomover (de Carvalho, 2011).

A implementação da Tarifa Zero universal nos transportes coletivos tem
sido proposta como uma possível solução para a promoção do direito à
cidade, como uma forma de mitigar as desigualdades características do
contexto urbano capitalista. Partidos, ativistas, movimentos sociais e
pesquisadores do tema reivindicam a luta pela Tarifa Zero enquanto
política de promoção não só do direito à cidade, mas também de
alternativas ecológicas e mais saudáveis de funcionamento da mobilidade
urbana em geral.

Hoje, estima-se que mais de 110 cidades adotam essa política, e a
discussão sobre as condições para a sua implementação tem ganhado
destaque no debate público. Dentre as regiões, a Sudeste é a que mais
demonstra adesão em relação à tarifa zero no transporte público.

\section{Objetivos}\label{objetivos}

Essa pesquisa tem como objetivo geral estabelecer um panorama sobre a
difusão da implementação da Tarifa Zero nos sistemas de transporte
público dos municípios brasileiros. Além disso, também serão analisadas
a correlação entre as variáveis Competição Política e PIB per capita com
a adoção dessa política.

Nesse sentido, testamos as seguintes hipóteses: - Municípios com alta
competitividade política (relacionado às eleições do executivo
municipal) estão mais propensos à adoção devido à pressão para se
diferenciar dos outros candidatos. - Municípios com alto PIB per capita
estão mais propensos à adoção, pois essa medida é utilizada como um
proxy de desenvolvimento econômico do município.

\section{Justificativa}\label{justificativa}

É evidente a relevância do transporte público no funcionamento das
cidades e na vida de seus habitantes, sendo um serviço essencial para a
garantia de uma cidadania plena à medida que o acesso à mobilidade
urbana interfere no acesso a outros direitos e serviços. Nesse sentido,
impossibilitar a utilização do serviço de transporte público por meio da
cobrança de tarifas diretas incompatíveis com a renda de determinados
grupos sociais significa reproduzir segregações socioespaciais e
intensificar as distintas formas de desigualdade que perpassam o
contexto urbano.

Como já apresentado, segundo a base de dados de Daniel Santini, cerca de
58\% dos municípios brasileiros que adotam sistemas de gratuidade
universal no transporte coletivo o fizeram a partir de 2020, abrangendo
cidades de diferentes tamanhos e localizadas em todas as 5 regiões do
Brasil. Segundo Santarém (2023), essa pauta já ganhou centralidade na
política institucional de grandes capitais do país e recebeu relativa
atenção nacional, não só pelo seu benefício social e ambiental, mas
também em virtude da crise de financiamento do transporte público que
atinge parte significativa dos municípios (intensificada pela redução de
passageiros graças à pandemia de COVID-19).

Isso posto, é de grande relevância estabelecer um panorama que analise o
histórico de implementação dessa política e que entenda se determinadas
características (competição política e PIB per capita) são relevantes
nesse processo.

\section{Método}\label{muxe9todo}

Para a realização desse trabalho, foram utilizados os dados compilados
por Daniel Santini (2024), referentes aos municípios que adotam a Tarifa
Zero universal no Brasil. Além disso, também foi utilizada uma base de
dados fornecida por Denilson Bandeira Coêlho (2016), a qual contém dados
sobre variáveis específicas de 5.564 municípios brasileiros. Importante
destacar que esses dados não estão atualizados, tendo como limite o ano
de publicação do artigo do autor citado.

Foi realizada uma mescla das duas bases de dados e o dataset foi
adaptado para essa política em questão, excluindo variáveis que não
seriam interessantes e incluindo novas.

Para a testagem das hipóteses sobre a relevância das variáveis de
competição política e de PIB per capita, utilizamos a regressão
logística por meio do R, com o intuito de medir possíveis correlações
com a variável dependente (adoção ou não da Tarifa Zero). A competição
política foi medida a partir da média de resultados eleitorais das
prefeituras e, também, do número efetivo de partidos. Nesse processo,
foram analisados apenas os dados de municípios com uma população entre
20 mil e 100 mil habitantes (totalizando 1370 municípios), em um esforço
de tornar o modelo mais equilibrado e, também, viabilizar o
processamento dos dados.

\section{Análise}\label{anuxe1lise}

\subsection{Contexto Nacional}\label{contexto-nacional}

\begin{verbatim}
## Warning: Expecting numeric in G1526 / R1526C7: got '#NULL!'
\end{verbatim}

\begin{verbatim}
## Warning: Expecting numeric in I1526 / R1526C9: got '#NULL!'
\end{verbatim}

\begin{verbatim}
## Warning: Using `size` aesthetic for lines was deprecated in ggplot2 3.4.0.
## i Please use `linewidth` instead.
## This warning is displayed once every 8 hours.
## Call `lifecycle::last_lifecycle_warnings()` to see where this warning was
## generated.
\end{verbatim}

\includegraphics{relatorioacademico_files/figure-latex/unnamed-chunk-2-1.pdf}
\#\# Municípios por porte

\includegraphics{relatorioacademico_files/figure-latex/unnamed-chunk-3-1.pdf}

\subsection{Regressão Logística}\label{regressuxe3o-loguxedstica}

\begin{verbatim}
## Analysis of Deviance Table (Type II tests)
## 
## Response: adotatz
##                    Df  Chisq Pr(>Chisq)   
## CompetiçãoPolítica  1 0.2627   0.608239   
## PIBpercapitaMedia   1 6.9388   0.008435 **
## ---
## Signif. codes:  0 '***' 0.001 '**' 0.01 '*' 0.05 '.' 0.1 ' ' 1
\end{verbatim}

\begin{itemize}
\tightlist
\item
  CompetiçãoPolítica: Não parece ter um efeito significativo na adoção
  da Tarifa Zero.
\item
  PIBpercapitaMedia: Tem um efeito significativo na adoção da Tarifa
  Zero (p = 0.00843). O coeficiente negativo sugere que, à medida que o
  PIB per capita médio aumenta, a probabilidade de adoção da Tarifa Zero
  diminui. Ou seja, municípios com PIB per capita mais alto são menos
  propensos a adotar a Tarifa Zero.
\end{itemize}

\subsection{Razões de chance}\label{razuxf5es-de-chance}

\includegraphics{relatorioacademico_files/figure-latex/gráfico razão de chances-1.pdf}

\section{Conclusões}\label{conclusuxf5es}

Com base na análise realizada, observamos que a competição política tem
um efeito positivo significativo na adoção da Tarifa Zero, indicando que
municípios com maior competição política são mais propensos a adotar
essa política. O PIB per capita não demonstrou um efeito tão
significativo quanto a competição política. Portanto, a hipótese de que
municípios com alta competição política são mais propensos a adotar a
Tarifa Zero foi confirmada. A política de Tarifa Zero é uma estratégia
adotada por municípios que buscam se destacar politicamente,
especialmente em um contexto de alta competição eleitoral.

A análise e os resultados obtidos fornecem uma visão importante sobre os
fatores que influenciam a implementação da Tarifa Zero e destacam a
necessidade de considerar o contexto político local ao avaliar políticas
públicas.

\end{document}
